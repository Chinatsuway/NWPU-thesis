\documentclass{qmes-polymer-properties}

\title{This is Report Title}
\studentname{Your Name}  % 设置学生姓名
\studentid{114514}    % 设置学号

\begin{document}
\maketitle

The following characters are for explanation. They should be deleted when you hand in this report.\cite{exampleref}

\section{Introduction}

15 marks.

This is the introduction section.

\section{Aims/Objection}

5 marks.

Each experiment should include the objective which is used to present the purpose of the experiment, the basic information you have and the skills you will acquire after the completion of the experiment.

\section{Materials and Methods}

15 marks.

List out the materials (e.g., appearance, color) used in the experiments. Describe the detailed operation procedure; the way data is collected.

Do not simply rewrite methods from the laboratory manual but note any changes that have been made.

Do not use personal pronouns and do write in the past tense and in paragraph. Stress the dos and don'ts in the whole procedure.

\section{Results}

25 marks.

This section puts figures which are drawn based on your tested data and analyze the figures. Describe all the important aspects of the experiment that you observe.

\textbf{You can combine the results section with discussion section.}

\section{Discussion}

25 marks.

This section is used to discuss your results. You may compare your results with
literatures (properly cited).

\begin{itemize}
    \item Determine what kind of polymer the test sample is, and give the judgment basis and source.
    \item Discuss the significance of the results to identify polymers.
    \item Discuss the discrepancies between theoretical and experimental results, and their likely causes.
\end{itemize}

\bibliographystyle{vancouver}
\bibliography{references}

\end{document}
