\documentclass{qmes-template}

\title{Report Template of QMES}
\course{QXU114514}
\date{\today}

\wordcount{1919810}

\addstudent{Zhang San}{20231234}
\addstudent{Li Si}{20245678}

\begin{document}

\maketitle
\tableofcontents

\section{Introduction}
This template is design for student of QMUL engineering school, NPU.

The main content of the document begins here. And page numbering starts here.

\section{Methodology}
Sample text for demonstrating the template features.

\subsection{User's customization}
For better use, the template has some customizable content.There is some
basically customization below:
\begin{lstlisting}
\title{Edit this to your report name}
\course{The course code}
\end{lstlisting}

Generally speaking, no changes are needed in this line. Unless you need the
report to show a date that's not today.
\begin{lstlisting}
\date{\today}
\end{lstlisting}

For the security reasons, users can only count the number of characters by
themselves. it's recommended that count your PDF files.
\begin{lstlisting}
\wordcount{The word count of your report}
\end{lstlisting}

For each team member, you need to add their personal information with a single
addstudent directive so that they can be displayed in the author column of the
title page
\begin{lstlisting}
\addstudent{your team parter's name}{his/her student number}
\end{lstlisting}

\newpage{}
\subsection{Figures and Tables}
Here we shows the example of Table and Figure. Notably, we fixed an issue where
the font size of the headers for legends and tables (and, of course, code
blocks) was incorrect.

\includecode[caption={The Implementing of the Table below}]{src/table.tex}

\begin{table}[h!]
    \centering
    \caption{Example of Table} 
    \begin{tabular}{ccc}
        \hline
        Animal & Food  & Size   \\
        \hline
        dog    & meat  & medium \\
        horse  & hay   & large  \\
        frog   & flies & small  \\
        \hline
    \end{tabular}
\end{table}

\includecode[caption={The Implementing of the figure below}]{src/figure.tex}

\begin{figure}[ht]
    \centering
    \includegraphics[width=0.2\textwidth]{./figure/LOGO.jpg}
    \caption{LOGO fig}
\end{figure}

\newpage{}
\subsection{Listings}
Example of implementing code blocks throught listings is below:

\includecode[caption={The Implementing of Code Block Below}]{src/main.tex}

\begin{lstlisting}[language=Python,caption={Example Code 2}] 
    def hello_world():
        print("Python code example") 
\end{lstlisting}

Here shows implement code blocks throught include code file:
\begin{lstlisting}[caption={The Implementing of Code Block Below}]
\includecode[language=C++,caption={Example for include code file}]{src/main.cpp}
\end{lstlisting}
% Include external code file
\includecode[language=C++,caption={Example for include code file}]{src/main.cpp}

\newpage{}
\section{Customization of Template}
You can futher customize your document by modifying some parameters in the $qmes-template.cls$.

\subsection{Caption Design}
Here, users can customize the font of the title in the floating environment (such as in the chart), as shown below. The default title font in the template is one size smaller than the main text font. In addition to the already displayed $font$, there are also parameters such as $margin$ and $labelfont$ that can be set.
\includecode[caption={Settings of Caption in \.cls File}]{src/caption.tex}

\subsection{Hyperlink and References Design}
The $hyperref$ package defines the colors of hyperlinks and reference symbols in an article. For the simplicity of the interface, the template is set to black by default.
\includecode[caption={Setting of Hyperlinks and Cites}]{src/hyperref.tex}

\subsection{Footnote Design}
In the shown code, users can conveniently modify the top and bottom of the page.
\includecode[caption={The Head and Footnote Settings in \.cls File}]{src/fancyhf.tex}

\section{Future Development}
We plan to develop Chinese version for NWPU.
\end{document}
